\documentclass[12pt]{article}
\usepackage{datetime}
\usepackage{acronym}
\usepackage{xcolor}
\setlength {\marginparwidth }{2cm}
\usepackage{todonotes}
\usepackage{hyperref}
% \usepackage{textcomp}
% \usepackage{mathrsfs}  % mathscr font
% \usepackage[colorlinks, filecolor=dark_blue, urlcolor=dark_blue, linkcolor=black, citecolor=black]{hyperref}

\usepackage[backend=biber]{biblatex}
\addbibresource{bibliography.bib}
\usepackage{cleveref}

\newcommand{\meta}[1]{{\color{blue}#1}}  

\newdateformat{monthyeardate}{\monthname[\THEMONTH], \THEYEAR}

\acrodef{iot}[IoT]{Internet of Things}
\acrodef{cps}[CPS]{Cyber-Physical System}
\acrodef{ac}[AC]{Aggregate Computing}
\acrodef{qos}[QoS]{Quality of Service}
\acrodef{cas}[CAS]{Collective Adaptive Systems}
\acrodef{api}[API]{Application Programming Interface}
\acrodef{iac}[IaC]{Infrastructure as Code}
\acrodef{ci}[CI]{Continuous Integration}
\acrodef{ai}[AI]{Artificial Intelligence}

\begin{document}

\begin{titlepage}
	\centering

	\textsc{\Large Ph.D Programme in Computer Science and Engineering}\\[0.5cm]
	\textsc{\Large Admission XXXIX Cycle}\\[0.6cm]

	\hrule width \hsize \kern 1mm \hrule width \hsize height 2pt
	\vspace{0.8cm}

	{\large \bfseries Research Project Proposal}\\[0.6cm]
	{\large \emph{Title of the Project}}\\[0.6cm]

	{\bfseries{\monthyeardate\today} \hfill \bfseries{Nicolas Farabegoli}}\\[0.6cm]

	\hrule width \hsize height 2pt \kern 1mm \hrule width \hsize height 1pt
	\vspace{0.4cm}

	\begin{abstract}
		% Da riscrivere in alcuni punti
		In recent years,
		the emergence of \ac{cps} has engendered a noteworthy surge in complexity and heterogeneity
		within the underlying infrastructure supporting these systems.
		Notably, the interplay between cloud, fog, and edge computing exemplifies the intricacy inherent in such systems.
		%
		Modern collective adaptive applications like \ac{iot}, human enhanced by wearable devices,
		swarm robotics, smart cities,
		are designed to be executed on several devices and to be deployed in
		heterogeneous infrastructures, ranging from cloud servers to wearable devices.
		%
		The availability of such a wide range of devices and infrastructures opens from one side
		better exploitation of the available resources and performance,
		but on the other side,
		it introduces complexity in the design and deployment of such applications.
		%
		This research project proposes to produce a framework for the design and deployment of
		collective adaptive applications on heterogeneous infrastructures.
		%
		Reconfiguration aspects will be considered,
		allowing the application to adapt to the changes in the infrastructure and external conditions.
		%
		The framework can leverage machine learning techniques to manage the complex task of reconfiguration
		to opportunistically balance the performance and energy consumption.
		%
		Since the \ac{iot} consists of a large number of devices,
		they produce a large amount of data from extremely scattered data sources.
		%
		The management of such data is a complex and delicate task,
		which will be addressed with state-of-the-art techniques.
	\end{abstract}
\end{titlepage}

\section{State of the Art}\label{sec:state-of-the-art}

\paragraph{Collective Adaptive Systems}
A \ac{cas} is a system composed of several entities that interact with each other and can dynamically adapt to changing environments or requirements~\cite{DBLP:conf/birthday/HolzlW11}.
%
Often,
the entities of these systems have their own goals and objectives,
and interactions with other entities or may lead to unexpected phenomena.
%
Many modern systems are collective adaptive systems: \emph{smart cities}, \emph{collective cyber-physical systems}, \emph{robot swarms} and \emph{sensor network}.
%
The notion of \ac{cas} was elaborated in the workshop ``Fundamentals of Collective Adaptive Systems''~\cite{DBLP:journals/corr/abs-1108-5643}
organized by the European Commission in 2009,
and in the following years,
it has been the subject of several research projects~\cite{DBLP:journals/procedia/ZambonelliCFMRSRTDSYNOMVFMW11,DBLP:series/lncs/8998}.

Coordination in \ac{cas} is a fundamental aspect to consider when adaptation properties are required.
%
Coordination~\cite{DBLP:journals/csur/Ciancarini96} represents an effective way to achieve a global goal in a distributed system.
%
When entities of the system start to communicate with each other,
the need for coordination arises.
%
In the literature,
there are several approaches to coordination~\cite{DBLP:journals/csur/Ciancarini96}:
\emph{message passing}, \emph{tuple space models}~\cite{DBLP:books/sp/omicini01/RossiCD01}, \emph{stigmergy models}~\cite{DBLP:journals/cogsr/Heylighen16}.

In several pervasive systems,
the traditional approaches in the engineering of these systems are not suitable,
due to the lack of mechanisms to address complex behaviour and limited scalability.

Situatedness and time awareness are two important aspects of pervasive systems.
%
For this reason,
spatial computing has emerged to manage these aspects as a first-class citizen~\cite{Beal_Viroli_2015}.
%
This approach makes it possible to obtain self-adaptive properties in the system,
and react to changes in the environment and faults conditions.

\ac{ac}~\cite{DBLP:journals/computer/BealPV15} is a novel approach that consists in manipulating computational fields,
in a declarative and compositional way.
%
This approach shifts the focus from a local perspective to a global one,
where a group of devices can be seen as a single entity in which the aggregate program is executed.
%
The main abstraction on which \ac{ac} is based is the \emph{computational field} i.e.,
a distributed data structure in which each device is mapped to a computational value.

One of the main benefits of this approach is the possibility to focus on high-level aspects and characteristics of the system,
without worrying about low-level details like communication, failure, distribution, and so on.
%
All the low-level aspects are managed by the underlying platform which manages the deployment and execution of the aggregate program.

% \ac{ac} is based on the \emph{field calculus}~\cite{DBLP:journals/tocl/AudritoVDPB19},
% a formal model that provides a set of primitives to manipulate computational fields.
% %
% In the field calculus,
% it is possible to define a collective behaviour via algorithms expressed as a computation of computational fields.
% %
% This enables the creation of a stack of building blocks,
% based on a resilient constructor that exposes \ac{api} used by the upper blocks.
% %
% In this way,
% if high-level building blocks depend on self-adaptive constructors,
% the former inherits the properties of the latter.
% %
% ScaFi~\cite{DBLP:journals/softx/CasadeiVAP22}, Protelis~\cite{DBLP:conf/sac/PianiniVB15} and FCPP~\cite{DBLP:conf/acsos/Audrito20} are the reference platforms for \ac{ac}.

% Currently, most of the experiments are performed in a simulated environment,
% where the devices are emulated.
% %
% Alchemist~\cite{DBLP:journals/jos/PianiniMV13} is a bio-inspired simulator that can be used to simulate \ac{ac} systems.
% %
% At the time of writing,
% there are no real-world, consolidated experiments that use \ac{ac};
% for this reason,
% closing the gap between the simulated and real-world environment is one of the open challenges of this research field.

\paragraph{Pulverisation}
To tackle the complexity of the deployment of \ac{cas},
several approaches have been proposed in the literature.
%
Some examples are: \emph{Osmotic Computing}~\cite{DBLP:journals/tiot/NehaPSSG22}, \emph{multi-tier programming}~\cite{DBLP:journals/csur/WeisenburgerWS20}.

\emph{Pulverisation}~\cite{DBLP:journals/fi/CasadeiPPVW20} represents an approach to distributed application partitioning and deployment.
%
Its goal is to provide a way to specify the functional semantics of the software in a deployment-independent way.
%
To do so,
the application logic should be designed considering a logical system,
which is a set of logical devices forming an arbitrary network topology.
%
The application of each logical device is decomposed into an ensemble of components,
representing respectively a set of \emph{sensors},
a set of \emph{actuators},
a \emph{state},
a \emph{communication} component and a \emph{computation} component modelling the behaviour of the device.
%
Decomposition of an application via pulverisation can be achieved in two ways:
either the application is designed with pulverization in mind,
or the application is developed using a framework supporting automatic decomposition such as \ac{ac} frameworks.
%
Once the application is pulverized,
a mapping must be provided between the logical system and the available infrastructure.
%
In this way,
a single logical device can be executed on multiple physical devices,
and a single physical device can execute multiple logical device components.
%
Moreover,
with no changes,
the application can be deployed on different infrastructures,
without the need to rewrite the application logic.
%
Even if the idea behind pulverisation is quite simple,
several challenges at different levels must be addressed to make it a reality:
\emph{communication}, \emph{portability}, \emph{runtime} are only some of them.

% --------------------
% Project description
% --------------------

\section{Project description}\label{sec:project-description}

With this project,
we aim to investigate opportunities and challenges in the design and deployment of
\ac{cas} on new and emergent infrastructures to opportunistically exploit them.
%
We identify in \ac{iot}, swarm robotics, crowds of wearable-augmented humans, and so on,
relevant application domains for this project.
%
Cloud computing has disrupted the way we think about computation and storage,
moving from a static and centralized model to a dynamic and distributed one.
%
Recently,
the emergence of a new paradigm called \emph{edge-cloud continuum}~\cite{DBLP:journals/iot/BittencourtISFM18}
evolved the cloud computing model to a more distributed one,
where several intermediate network layers and data processing patterns exist between the cloud data center and devices.
%
This infrastructure is characterized by a multi-layer heterogeneous infrastructure,
ranging from powerful cloud servers to small connected devices.
%
Nevertheless,
the design and deployment of \ac{cas} on this infrastructure is not trivial;
several implications must be considered to effectively exploit the infrastructure's full potential,
resulting in an open research problem.

With this project,
we pursue one of the most promising approaches to tackle this complexity,
by leveraging the \emph{Pulverisation}~\cite{DBLP:journals/fi/CasadeiPPVW20} aproach.
%
This approach tries to address the complexity by neatly separating the business logic of the system from infrastructural aspects,
letting to focus on the functional aspects of the systems and delaying non-functional and infrastructural aspects.
%
The idea is to partition the software system into five components (\emph{sensors}, \emph{actuators}, \emph{behavior}, \emph{communication} and \emph{state})
that can be independently deployed on the available infrastructure.
%
One current limitation of this approach regards the runtime reconfiguration of the system based on the changing conditions or requirements.
%
This project aims to extend the current pulverisation approach to support dynamic reconfiguration of the system
by combining two approaches: \emph{Rule-based} and \emph{AI-based reconfiguration}.
%
The former can be used to define a set of static rules that can be used to reconfigure the system based on pre-determined and well-known conditions.
%
The latter can be extremely useful in contexts where the unpredictable conditions can be extremely high,
or where many different factors must be considered to determine the best configuration for the system.
%
The ability to dynamically reconfiguration is a fundamental aspect,
for example,
in contexts where energy-efficient systems are required,
or where the balance between consumption and performance can be strategic.

One other goal of this project is to bridge the gap between the simulation and the real world,
providing a set of \emph{methodology}, \emph{tools}, and \emph{techniques}
that can be seamlessly exploited to effectively tackle the system's complexity.

\ac{cas} can be composed of several different devices that can produce a large amount of data.
%
This kind of data can be characterized by many formats, different sources, and different frequencies.
%
The management of such data can be compicated and can lead to a large amount of data that cannot be processed with traditional techniques.
%
For this reason,
this project aims to provide a new approach to the engineering of data sources where a heterogeneous infrastructure is involved.
%
In this way,
the data can be available to be processed using big-data analysis techniques,
opening to new opportunities in the context of \ac{cas}.

As follows, is provided a brief overview of the main activities that will be carried out.

\subsection{Activities}\label{subsec:activities}

\paragraph{Deployment via pulverisation}
The first fundamental step of this project is to provide an implementation for
the pulverisation approach via a dedicated framework.
%
The framework will be designed to be modular and extensible to allow the integration
of different technologies and give the possibility to be used in conjunction with
\ac{ac} frameworks.
%
The framework should give the possibility to specify the system,
use the system specification to simulate the system
and finally, with no changes in the code base,
deploy the system on real devices.

\paragraph{Dynamic reconfiguration and relocation}
Another key contribution of this project is to support dynamic reconfiguration of the system,
allowing it to adapt to the changes in the infrastructure and external conditions.
%
In the original model of the pulverisation approach,
there is no support for dynamic reconfiguration of the system.
%
For this reason,
the project will investigate the extension of the pulverisation approach
to support dynamic reconfiguration and relocation of the system inside the framework stack.
%
To do so,
the project will leverage machine learning techniques to manage the complex task of reconfiguration
to opportunistically balance the performance and energy consumption or any other system-specific \ac{qos}.

\paragraph{Bind AC framework to pulverisation}
The need to integrate \ac{ac} framework into the pulverisation one is
a fundamental step to provide a complete solution for the design and deployment of
collective adaptive applications on heterogeneous infrastructures.
%
Indeed, combining the pulverisation approach with \ac{ac} frameworks
allows to describe the system's behavior at a macro-level
but at the same time exploiting the pulverisation to
effectively deploy the system on the available infrastructure,
managing also the dynamic reconfiguration of the system transparently and seamlessly.

\paragraph{Big-data for Aggregate Computing}
The investigation of big-data techniques in the context of \ac{cas} can be an interesting research topic.
%
The huge amount of data produced by \ac{iot} requires new techniques and approaches to be effectively managed.
%
The project will investigate the use of big-data techniques in the context of \ac{cas} and \ac{iot} systems,
to enrich the current \ac{ac} model with data-analysis capabilities.

\subsection{Scope}\label{subsec:scope}

\paragraph{Software engineering}
The challenges of the edge-cloud continuum lead to the need for new models,
design techniques, algorithms, languages, and tools to manage the complexity
of \ac{cps}.
%
With this new design approach,
there is the need to have the possibility to test and validate the system via simulation
and then seamlessly moves to the real deployments with a small effort.
%
This kind of seamless integration between simulation and real deployment can be strategic
to simplify the validation of the system and to reduce the time to market.

\paragraph{Distributed intelligence}
\ac{ai} is a computer science branch that has met big improvements over the last few years.
%
Even though the \ac{ai} usually tries to solve vertical and specific problems,
making it difficult to be applied in contexts like \ac{cas},
which are highly distributed by nature.
%
The value of a distributed intelligence leveraging \ac{ai} techniques can be strategic in \ac{cas},
in particular for adapting -- intelligently -- the system to changing conditions or to maintain certain \ac{qos}.


\paragraph{Energy and Cost efficiency}
One of the current trends in the IT industry is the \emph{green computing}.
%
The need to reduce energy consumption and consequently the carbon footprint of the systems can be a requirement in many future systems.
%
For this reason,
supporting the design and implementation of systems in which from the many \ac{qos} parameters,
energy consumption is one of the most important,
must be a future direction of \ac{cps} and \ac{iot} systems.

\paragraph{Cyper-physical Systems}
We refer to \ac{cps} as a system that is composed of physical and computational components
that are deeply intertwined and interact with each other.
%
\emph{Smart cities}, \emph{Smart home}, \emph{Precision agricolture} are just some examples of \ac{cps}.
%
All of these systems can be very different from each other,
but they share some characteristics and challenges,
for example: \emph{dynamicity}, \emph{volume} (devices and produced data) and \emph{heterogeneity}.
%
The project can be framed in the context of \ac{cps} by providing an effective general solution to the aforementioned challenges,
which can be tailored to the specific needs of the system.
%
Moreover,
even the \ac{cas} can take advantage of the opportunistic deployment and adaptation of the system,
which is a key point of this project.

\subsection{Technology}\label{subsec:technology}

\paragraph{Cloud computing}
One of the main objectives of the pulverisation approach is to simplify the development
and the deployment of complex systems on heterogeneous infrastructures.
%
The edge-cloud continuum is a new emerging paradigm that allows exploiting the
advantages of both cloud and edge computing (and its continuum).
%
For this reason, it is fundamental to have a deep understanding of the cloud computing
paradigm and its technologies to provide a complete solution for the design and deployment
of this kind of system.
%
In recent years, several cloud providers have emerged, and each of them provides
a set of services and technologies to support the development of cloud-based applications.
%
\emph{Google Cloud Platform}~\footnote{\url{https://cloud.google.com/}},
\emph{Amazon Web Services}~\footnote{\url{https://aws.amazon.com/}},
and \emph{Microsoft Azure}~\footnote{\url{https://azure.microsoft.com/}}
are the most popular cloud providers, and they will be the main focus of this project.
%
Could be strategic to enable the framework to be used with different cloud providers
to allow the user to choose the one that best suits his needs.
%
In recent years,
the way to provision infrastructure has changed.
%
The trend is to fully automate the provisioning of the infrastructure via dedicated tools,
like Terraform~\footnote{\url{https://www.terraform.io/}} and Ansible~\footnote{\url{https://www.ansible.com/}}.
%
These tools fall in the category of \ac{iac}: they allow to describe the infrastructure in a machine-readable format
instead of using interactive configuration tools.
%
Moreover,
these tools provide a set of \ac{api} to interact with them programmatically,
allowing the integration with other tools and frameworks.
%
As a consequence,
these tools can be used in a \ac{ci} pipeline to fully automate the deployment of the system.
%
\ac{iac} represents a powerful technology in the context of the edge-cloud continuum,
allowing a dynamic, programmable and automated deployment of the system on the available infrastructure.

\paragraph{Big-data}
The \ac{iot} systems consist of a large number of devices that produce a large amount of data
from extremely scattered data sources.
%
The handling of such data requires an adequate infrastructure and tools to manage them.
%
For this reason, the project will investigate the use of big-data technologies
such as \emph{Apache Spark}~\footnote{\url{https://spark.apache.org/}},
\emph{Apache Flink}~\footnote{\url{https://flink.apache.org/}} and their integration with the pulverisation approach and the \ac{ac} frameworks.
%
The union of big-data technologies and the pulverisation approach can open new opportunities for \ac{cas}.

\paragraph{Distributed technology}
The resilience of the system is a fundamental aspect to consider when designing
and deploying a collective adaptive application.
%
Due to the highly distributed nature of the \ac{iot} and \ac{ac} systems,
the failure of a node is the norm rather than the exception.
%
For this reason, will be investigated the use of distributed technologies like \emph{MOM},
specifically, RabbitMQ~\footnote{\url{https://www.rabbitmq.com/}} and ZeroMQ~\footnote{\url{https://zeromq.org/}};
microservices patterns and approaches to provide a resilient system.
%
Orchestration tools like Kubernetes~\footnote{\url{https://kubernetes.io/}} can be used to manage the deployment of the system
and the dynamic reconfiguration of it.

\section{Expected results}\label{sec:expected-results}

\subsection{One year goal}

\paragraph{Pulverization framework}
Realize an effective framework to support the development of complex systems
via the pulverisation approach.
%
We will try to provide a modular and extensible framework to allow the integration
of different technologies and give the possibility to be used in conjunction with
\ac{ac} frameworks.

\paragraph{Dynamic reconfiguration and adaptations}
Extend the pulverisation approach to support dynamic reconfiguration and relocation
of the system inside the framework stack.
%
To do so,
we will investigate the use of state-of-the-art distributed technologies
improving the resilience of the system opportunistically exploiting
the underlying infrastructure.
%
Firstly we will focus on rule-based reconfiguration,
in particular device-based rules, and then we will investigate
global rules to manage the system's behavior at a macro level.

\subsection{PhD goal}

\paragraph{AI-based system reconfiguration}
\meta{
Opportunistic and flexible deployment can take a big advantage
from \ac{ai} techniques to intelligently adapt the system to specific \ac{qos} requirements.
%
This kind of ``intelligent reconfiguration'' can be used to manage scenarios
where computational power and energy consumption are the main concerns,
but the complexity of the system can be difficult to manage.
}

\paragraph{Combine pulverization with AC}
Provide a solution that integrates \ac{ac} frameworks with the pulverisation approach.
%
In this way,
we provide a possible flexible and effective deployment solution for systems
where the \ac{ac} paradigm is used to implement the self-organization logic of the system.

\paragraph{Big-data for Aggregate Computing}
\meta{
Use of big-data technologies to manage the data produced by the system
so that new sorts of knowledge can be extracted from the system.
}

\paragraph{Application and scenarios for this new approach}
Creation of significant applications and services in the context of \emph{smart cities and home},
\emph{smart factories}, \emph{large-scale \ac{iot} systems}, \emph{swarm of drones} and so on,
to ground the proposed approaches and demonstrate their effectiveness.

\subsection{Long term contributions}
\meta{
With the proposed project, we aim to close the gap between the simulation of \ac{cas} and the real world.
%
Moreover, we aim to provide an effective ecosystem -- composed of methodologies, tools, and techniques --
to support the design and deployment of \ac{cas} and \ac{iot} systems on the edge-cloud continuum.

With this project,
we expect to provide one of the first concrete solutions for the engineering of the above-mentioned systems,
and we expect to greater understand the challenges and opportunities of this new approach,
but also the role it can play in modern scenarios like smart cities, smart factories, large-scale \ac{iot} systems, and so on.

In conclusion,
this project can be intended as a first step toward the creation of solid support
for the engineering of resource-efficient, resilient, and adaptive systems
which can exploit the resources also in a sustainable way.
}

\newpage

\printbibliography

\end{document}
