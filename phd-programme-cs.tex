\documentclass[12pt]{article}
\usepackage{datetime}
\usepackage{acronym}
\usepackage{xcolor}
% \usepackage{textcomp}
% \usepackage{mathrsfs}  % mathscr font
% \usepackage[colorlinks, filecolor=dark_blue, urlcolor=dark_blue, linkcolor=black, citecolor=black]{hyperref}

\usepackage{cleveref}

\newcommand{\meta}[1]{{\color{blue}#1}}  

\newdateformat{monthyeardate}{\monthname[\THEMONTH], \THEYEAR}

\acrodef{iot}[IoT]{Internet of Things}
\acrodef{cps}[CPS]{Cyber-Physical System}
\acrodef{ac}[AC]{Aggregate Computing}

\begin{document}

\begin{titlepage}
	\centering

	\textsc{\Large Ph.D Programme in Computer Science and Engineering}\\[0.5cm]
	\textsc{\Large Admission XXXIX Cycle}\\[0.6cm]

	\hrule width \hsize \kern 1mm \hrule width \hsize height 2pt
	\vspace{0.8cm}

	{\large \bfseries Research Project Proposal}\\[0.6cm]
	{\large \emph{Title of the Project}}\\[0.6cm]

	{\bfseries{\monthyeardate\today} \hfill \bfseries{Nicolas Farabegoli}}\\[0.6cm]

	\hrule width \hsize height 2pt \kern 1mm \hrule width \hsize height 1pt
	\vspace{0.4cm}

	\begin{abstract}
		% Da riscrivere in alcuni punti
		In recent years,
		the emergence of \ac{cps} has engendered a noteworthy surge in complexity and heterogeneity
		within the underlying infrastructure supporting these systems.
		Notably, the interplay between cloud, fog, and edge computing exemplifies the intricacy inherent in such systems.
		%
		Modern collective adaptive applications like \ac{iot}, swarm robotics, and smart cities,
		are designed to be executed on several devices and to be deployed in
		heterogeneous infrastructures, ranging from cloud servers to wearable devices.
		%
		The availability of such a wide range of devices and infrastructures opens from one side
		better exploitation of the available resources and performance,
		but on the other side,
		it introduces complexity in the design and deployment of such applications.
		\meta{
		This research project proposes to produce a framework for the design and deployment of
		collective adaptive applications on heterogeneous infrastructures,
		where the \ac{ac} can be injected into.
		%
		Reconfiguration aspects will be considered,
		allowing the application to adapt to the changes in the infrastructure and external conditions.
		%
		The framework can leverage machine learning techniques to manage the complex task of reconfiguration
		to opportunistically balance the performance and energy consumption.
		%
		Since the \ac{iot} consists of a large number of devices,
		they produce a large amount of data from extremely scattered data sources.
		%
		The management of such data is a complex and delicate task,
		which will be addressed with state-of-the-art techniques.
		}
	\end{abstract}
\end{titlepage}

\section{Introduction}\label{sec:introduction}

\section{State of the Art}\label{sec:state-of-the-art}

\section{Project description}\label{sec:project-description}

\section{Expected results}\label{sec:expected-results}

\end{document}
