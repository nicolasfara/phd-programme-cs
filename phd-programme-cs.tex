\documentclass[12pt]{article}
\usepackage{datetime}
\usepackage{acronym}
\usepackage{xcolor}
\setlength {\marginparwidth }{2cm}
\usepackage{todonotes}
% \usepackage{textcomp}
% \usepackage{mathrsfs}  % mathscr font
% \usepackage[colorlinks, filecolor=dark_blue, urlcolor=dark_blue, linkcolor=black, citecolor=black]{hyperref}

\usepackage{cleveref}

\newcommand{\meta}[1]{{\color{blue}#1}}  

\newdateformat{monthyeardate}{\monthname[\THEMONTH], \THEYEAR}

\acrodef{iot}[IoT]{Internet of Things}
\acrodef{cps}[CPS]{Cyber-Physical System}
\acrodef{ac}[AC]{Aggregate Computing}
\acrodef{qos}[QoS]{Quality of Service}
\acrodef{cas}[CAS]{Collective Adaptive Systems}

\begin{document}

\begin{titlepage}
	\centering

	\textsc{\Large Ph.D Programme in Computer Science and Engineering}\\[0.5cm]
	\textsc{\Large Admission XXXIX Cycle}\\[0.6cm]

	\hrule width \hsize \kern 1mm \hrule width \hsize height 2pt
	\vspace{0.8cm}

	{\large \bfseries Research Project Proposal}\\[0.6cm]
	{\large \emph{Title of the Project}}\\[0.6cm]

	{\bfseries{\monthyeardate\today} \hfill \bfseries{Nicolas Farabegoli}}\\[0.6cm]

	\hrule width \hsize height 2pt \kern 1mm \hrule width \hsize height 1pt
	\vspace{0.4cm}

	\begin{abstract}
		% Da riscrivere in alcuni punti
		In recent years,
		the emergence of \ac{cps} has engendered a noteworthy surge in complexity and heterogeneity
		within the underlying infrastructure supporting these systems.
		Notably, the interplay between cloud, fog, and edge computing exemplifies the intricacy inherent in such systems.
		%
		Modern collective adaptive applications like \ac{iot}, swarm robotics, and smart cities,
		are designed to be executed on several devices and to be deployed in
		heterogeneous infrastructures, ranging from cloud servers to wearable devices.
		%
		The availability of such a wide range of devices and infrastructures opens from one side
		better exploitation of the available resources and performance,
		but on the other side,
		it introduces complexity in the design and deployment of such applications.
		\meta{
		This research project proposes to produce a framework for the design and deployment of
		collective adaptive applications on heterogeneous infrastructures,
		where the \ac{ac} can be injected into.
		%
		Reconfiguration aspects will be considered,
		allowing the application to adapt to the changes in the infrastructure and external conditions.
		%
		The framework can leverage machine learning techniques to manage the complex task of reconfiguration
		to opportunistically balance the performance and energy consumption.
		%
		Since the \ac{iot} consists of a large number of devices,
		they produce a large amount of data from extremely scattered data sources.
		%
		The management of such data is a complex and delicate task,
		which will be addressed with state-of-the-art techniques.
		}
	\end{abstract}
\end{titlepage}

\section{Introduction}\label{sec:introduction}

Emerging application,
like \ac{cas},
are characterized by a large number of devices and agents
that interact to solve problems or provide services,
while adapting as a whole to changes in the environment.
%
Some examples are \ac{iot}, swarm robotics, crowds of wearable-augmented humans, and so on.
%
The infrastructures on which these applications are deployed may vary,
but in the last years, it is mounting interest in the idea of \emph{edge-cloud continuum}.
%
This infrastructure is characterized by a multi-layer heterogeneous infrastructure,
ranging from powerful cloud servers to small connected devices.

The edge-cloud systems are extremely valuable for \ac{cas},
which generally feature components that, depending on the conditions,
could benefit from being deployed on different layers of the infrastructure.
%
Often,
applications are deployed with a specific infrastructure in mind,
resulting in a business logic that is tightly coupled with the infrastructure.
%
The higher the coupling between the application and the infrastructure,
the harder it is for the former to exploit the latter's full potential and adapt to changes.

The \emph{pulverisation} represents a promising approach to tackle this problem.
%
This approach partitions the software system into five components that can be independently deployed on the available infrastructure
by decoupling the business logic from deployment concerns.
%
One current limitation of this approach regards the runtime reconfiguration of the system based on the changing conditions or requirements.
%
In this regard,
AI/ML techniques can be leveraged to manage the complexity of the reconfiguration process,
by determining a sort of ``oracle'' that can decide the best configuration for the system based on the current conditions.

One modern and promising way of engineering \ac{cas} is the \ac{ac} paradigm.
%
\ac{ac} is a novel approach that consists in manipulating computational fields,
in a declarative and compositional way.
%
The direction of this project is to investigate the use of \ac{ac} to engineer \ac{cas},
where the \emph{pulverisation} simplify the deployment in the \emph{edge-cloud continuum}.

Nowadays,
the ability to process huge volumes of data is an important aspect when complex data analysis can be performed.
%
The \ac{iot} systems are characterized by a large number of devices that produce a large amount of data,
and for this reason,
the management and processing of such data cannot be performed with traditional techniques.
%
Another direction of this project is to investigate the use of state-of-the-art techniques to manage the data produced by \ac{iot} or \ac{cps}
leading to a new, unexplored research direction: \emph{Big Data in \ac{ac}}.

This manuscript is organized as follows:
\Cref{sec:state-of-the-art} presents the state of the art of the research field,
\Cref{sec:project-description} describes what are the main activities of the work,
the technologies involved and the main scopes.
%
The last section, \Cref{sec:expected-results}, concludes the manuscript with short-term and long-term achievement.


\section{State of the Art}\label{sec:state-of-the-art}

\section{Project description}\label{sec:project-description}

With this project,
we aim to investigate opportunities and challenges in the design and deployment of
\ac{cas} on new and emergent infrastructures to opportunistically exploit them.
%
Cloud computing has disrupted the way we think about computation and storage,
moving from a static and centralized model to a dynamic and distributed one.
%
Recently,
the emergence of a new paradigm called \emph{Edge computing}
evolved the cloud computing model to a more distributed one,
where several intermediate network layers and data processing patterns exist between the cloud data center and devices.
%
This new paradigm is characterized by a layered and heterogeneous infrastructure,
which can be opportunistically exploited to improve the performance of the applications.
%
Nevertheless,
the design and deployment of systems, in particular \ac{cas},
are not trivial tasks and are currently open research challenges.

With this project,
we pursue one of the most promising approaches to tackle this complexity,
by leveraging the \emph{Pulverisation} aproach.
%
This approach tries to address the complexity by neatly separating the business logic of the system from infrastructural aspects.

One of the ambitious goals of this project is to bridge the gap between the simulation and the real world,
providing a set of \emph{methodology}, \emph{tools}, and \emph{techniques}
that can be seamlessly exploited to effectively tackle the system's complexity.
%
As a consequence,
another fundamental goal is to extend the current pulverisation approach to support dynamic reconfiguration of the system.
%
This extension can be achieved by combining two approaches: \emph{Rule-based} and \emph{Machine Learning}.
%
The combination of these two approaches can lead to a more effective and efficient reconfiguration of the system,
to substantially improve the \ac{qos} of the system.
%
This is a fundamental aspect,
for example,
in contexts where energy-efficient systems are required,
or where the balance between consumption and performance can be strategic.

The current \ac{iot} systems are characterized by a large number of devices,
which produces a large amount of data from extremely scattered data sources.
%
The management of such data is a complex and delicate task,
but can open to new opportunities in the context of \ac{cas}.
%
For this reason,
we aim to investigate opportunities in the use of \emph{big data} techniques in the context of \ac{cas} and \ac{iot} systems.


% This project aims to figure out practical and effective methodology, tools and framework to tackle the complexity of
% collective adaptive applications face when deployed on heterogeneous infrastructures,
% and adaptation is a must-have requirement.
%
% In particular, the project will focus on finding methods and tools to support the
% engineering of complex adaptive systems mainly focused on the \ac{iot} domain
% where the \ac{ac} paradigm emerges as a prominent solution.

% Several challenges arise when designing
% and deploying collective adaptive applications on heterogeneous infrastructures;
% for this reason, most of the research efforts leverage the simulation
% as a powerful tool to test and validate the proposed solutions.
% %
% With this project, we aim to ``bridge the gap'' between the simulation and the real world,
% providing, as stated above, a set of \emph{methodology}, \emph{tools}, and \emph{techniques}
% that can be seamlessly exploited to effectively tackle the system's complexity.

% To achieve this goal, the starting point is the \emph{pulverisation} approach
% which born in the \ac{ac} framework and represents a powerful and effective 
% methodology to describe such systems via a compositional approach.
% %
% Extension to this approach will be investigated to support dynamic reconfiguration
% of the system, allowing it to adapt to the changes in the infrastructure and external conditions
% via, for example, machine learning techniques.
% %
% \emph{Variety} and \emph{volume} characterize the modern \ac{iot} systems,
% and the management of such data is a complex and delicate task,
% due to security implications and privacy concerns.
% %
% For this reason, the project will try to address these issues with
% a new concept of ``Big-data for Aggregate Computing'' by leveraging
% state-of-the-art big-data techniques.

As follows, is provided a brief overview of the main activities that will be carried out.

\subsection{Activities}\label{subsec:activities}

\meta{
\paragraph{Deployment via pulverisation}
The first fundamental step of this project is to provide an implementation for
the pulverisation approach via a dedicated framework.
%
The framework will be designed to be modular and extensible to allow the integration
of different technologies and give the possibility to be used in conjunction with
\ac{ac} frameworks.

\paragraph{Dynamic reconfiguration and relocation}
Another key contribution of this project is to support dynamic reconfiguration of the system,
allowing it to adapt to the changes in the infrastructure and external conditions.
%
In the original model of the pulverisation approach,
there is no support for dynamic reconfiguration of the system.
%
For this reason,
the project will investigate the extension of the pulverisation approach
to support dynamic reconfiguration and relocation of the system inside the framework stack.
%
To do so,
the project will leverage machine learning techniques to manage the complex task of reconfiguration
to opportunistically balance the performance and energy consumption or any other system-specific \ac{qos}.

\paragraph{Bind AC framework to pulverisation}
The need to integrate \ac{ac} framework into the pulverisation one is
a fundamental step to provide a complete solution for the design and deployment of
collective adaptive applications on heterogeneous infrastructures.
%
Indeed, combining the pulverisation approach with \ac{ac} frameworks
allows to describe the system's behavior at a macro-level
but at the same time exploiting the pulverisation to
effectively deploy the system on the available infrastructure,
managing also the dynamic reconfiguration of the system transparently and seamlessly.

\paragraph{Big-data for Aggregate Computing}
???
}

\subsection{Scope}\label{subsec:scope}

\paragraph{Software engineering}
TODO

\paragraph{Energy and Cost efficiency}
TODO

\paragraph{Collective Adaptive Systems}
TODO

\subsection{Technology}\label{subsec:technology}

\meta{
\paragraph{Cloud computing}
One of the main objectives of the pulverisation approach is to simplify the development
and the deployment of complex systems on heterogeneous infrastructures.
%
The edge-cloud continuum is a new emerging paradigm that allows exploiting the
advantages of both cloud and edge computing (and its continuum).
%
For this reason, it is fundamental to have a deep understanding of the cloud computing
paradigm and its technologies to provide a complete solution for the design and deployment
of this kind of system.
%
In recent years, several cloud providers have emerged, and each of them provides
a set of services and technologies to support the development of cloud-based applications.
%
\emph{Google Cloud Platform}, \emph{Amazon Web Services}, and \emph{Microsoft Azure}
are the most popular cloud providers, and they will be the main focus of this project.
%
Could be strategic to enable the framework to be used with different cloud providers
to allow the user to choose the one that best suits his needs.

\todo{IaaS tools and orchestration tools}


\paragraph{Big-data}
The \ac{iot} systems consist of a large number of devices that produce a large amount of data
from extremely scattered data sources.
%
The handling of such data requires an adequate infrastructure and tools to manage them.
%
For this reason, the project will investigate the use of big-data technologies
such as \emph{Apache Spark}, \emph{Apache Flink} and their integration with the pulverisation approach and the \ac{ac} frameworks.
%
The union of big-data technologies and the pulverisation approach can open new opportunities for \ac{cas}.

\paragraph{Distributed technology}
The resilience of the system is a fundamental aspect to consider when designing
and deploying a collective adaptive application.
%
Due to the highly distributed nature of the \ac{iot} and \ac{ac} systems,
the failure of a node is the norm rather than the exception.
%
For this reason, will be investigated the use of distributed technologies
like \emph{MOM}, microservices patterns and approaches,
and tools like \emph{Apche Zookeeper} to provide a resilient system.

\todo{Orchestration like K8s}
}

\section{Expected results}\label{sec:expected-results}

\subsection{One year goal}

\paragraph{Pulverization framework}
Realize an effective framework to support the development of complex systems
via the pulverisation approach.
%
We will try to provide a modular and extensible framework to allow the integration
of different technologies and give the possibility to be used in conjunction with
\ac{ac} frameworks.

\paragraph{Dynamic reconfiguration and adaptations}
Extend the pulverisation approach to support dynamic reconfiguration and relocation
of the system inside the framework stack.
%
To do so,
we will investigate the use of state-of-the-art distributed technologies
improving the resilience of the system opportunistically exploiting
the underlying infrastructure.
%
Firstly we will focus on rule-based reconfiguration,
in particular device-based rules, and then we will investigate
global rules to manage the system's behavior at a macro level.

\subsection{PhD goal}

\paragraph{Opportunistic deployment and reconfiguration}
\meta{
Flexible and opportunistic deployment can take a big advantage of the cloud
computing in conjunction with orchestration technologies.
%
In particular, these technologies can be used to manage scenarios where
computational power and energy consumption are the main concerns.
%
In this context, AI-based techniques can represent a promising
approach to intelligently adapt the system to specific \ac{qos} requirements.
}

\paragraph{Big-data for Aggregate Computing}
\meta{
Use of big-data technologies to manage the data produced by the system

}

\paragraph{Application and scenarios for this new approach}
\meta{TODO}

\subsection{Long term contributions}
\meta{
With the proposed project, we aim to close the gap between the simulation of \ac{cas} and the real world.
%
Moreover, we aim to provide an effective ecosystem -- composed of methodologies, tools, and techniques --
to support the design and deployment of \ac{cas} and \ac{iot} systems on the edge-cloud continuum.

With this project,
we expect to provide one of the first concrete solutions for the engineering of the above-mentioned systems,
and we expect to greater understand the challenges and opportunities of this new approach,
but also the role it can play in modern scenarios like smart cities, smart factories, large-scale \ac{iot} systems, and so on.

In conclusion,
this project can be intended as a first step toward the creation of solid support
for the engineering of resource-efficient, resilient, and adaptive systems
which can exploit the resources also in a sustainable way.
}

\end{document}
