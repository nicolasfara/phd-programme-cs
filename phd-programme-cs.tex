\documentclass[12pt]{article}
\usepackage{datetime}
\usepackage{acronym}
\usepackage{xcolor}
% \usepackage{textcomp}
% \usepackage{mathrsfs}  % mathscr font
% \usepackage[colorlinks, filecolor=dark_blue, urlcolor=dark_blue, linkcolor=black, citecolor=black]{hyperref}

\usepackage{cleveref}

\newcommand{\meta}[1]{{\color{blue}#1}}  

\newdateformat{monthyeardate}{\monthname[\THEMONTH], \THEYEAR}

\acrodef{iot}[IoT]{Internet of Things}
\acrodef{cps}[CPS]{Cyber-Physical System}
\acrodef{ac}[AC]{Aggregate Computing}
\acrodef{qos}[QoS]{Quality of Service}

\begin{document}

\begin{titlepage}
	\centering

	\textsc{\Large Ph.D Programme in Computer Science and Engineering}\\[0.5cm]
	\textsc{\Large Admission XXXIX Cycle}\\[0.6cm]

	\hrule width \hsize \kern 1mm \hrule width \hsize height 2pt
	\vspace{0.8cm}

	{\large \bfseries Research Project Proposal}\\[0.6cm]
	{\large \emph{Title of the Project}}\\[0.6cm]

	{\bfseries{\monthyeardate\today} \hfill \bfseries{Nicolas Farabegoli}}\\[0.6cm]

	\hrule width \hsize height 2pt \kern 1mm \hrule width \hsize height 1pt
	\vspace{0.4cm}

	\begin{abstract}
		% Da riscrivere in alcuni punti
		In recent years,
		the emergence of \ac{cps} has engendered a noteworthy surge in complexity and heterogeneity
		within the underlying infrastructure supporting these systems.
		Notably, the interplay between cloud, fog, and edge computing exemplifies the intricacy inherent in such systems.
		%
		Modern collective adaptive applications like \ac{iot}, swarm robotics, and smart cities,
		are designed to be executed on several devices and to be deployed in
		heterogeneous infrastructures, ranging from cloud servers to wearable devices.
		%
		The availability of such a wide range of devices and infrastructures opens from one side
		better exploitation of the available resources and performance,
		but on the other side,
		it introduces complexity in the design and deployment of such applications.
		\meta{
		This research project proposes to produce a framework for the design and deployment of
		collective adaptive applications on heterogeneous infrastructures,
		where the \ac{ac} can be injected into.
		%
		Reconfiguration aspects will be considered,
		allowing the application to adapt to the changes in the infrastructure and external conditions.
		%
		The framework can leverage machine learning techniques to manage the complex task of reconfiguration
		to opportunistically balance the performance and energy consumption.
		%
		Since the \ac{iot} consists of a large number of devices,
		they produce a large amount of data from extremely scattered data sources.
		%
		The management of such data is a complex and delicate task,
		which will be addressed with state-of-the-art techniques.
		}
	\end{abstract}
\end{titlepage}

\section{Introduction}\label{sec:introduction}

\section{State of the Art}\label{sec:state-of-the-art}

\section{Project description}\label{sec:project-description}

This project aims to figure out a practical and effective methodology to tackle the complexity
collective adaptive applications face when deployed on heterogeneous infrastructures,
and adaptation is a must-have requirement.
%
In particular, the project will focus on finding methods and tools to support the
engineering of complex adaptive systems mainly focused on the \ac{iot} domain
where the \ac{ac} paradigm emerges as a prominent solution.

Several challenges arise when designing
and deploying collective adaptive applications on heterogeneous infrastructures;
for this reason, most of the research efforts leverage the simulation
as a powerful tool to test and validate the proposed solutions.
%
With this project, we aim to ``bridge the gap'' between the simulation and the real world,
providing, as stated above, a set of \emph{methodology}, \emph{tools}, and \emph{techniques}
that can be seamlessly exploited to effectively tackle the system's complexity.

To achieve this goal, the starting point is the \emph{pulverisation} approach
which born in the \ac{ac} framework and represents a powerful and effective 
methodology to describe such systems via a compositional approach.
%
Extension to this approach will be investigated to support dynamic reconfiguration
of the system, allowing it to adapt to the changes in the infrastructure and external conditions
via, for example, machine learning techniques.
%
\emph{Variety} and \emph{volume} characterize the modern \ac{iot} systems,
and the management of such data is a complex and delicate task,
due to security implications and privacy concerns.
%
For this reason, the project will try to address these issues with
a new concept of ``Big-data for Aggregate Computing'' by leveraging
state-of-the-art big-data techniques.

As follows, is provided a brief overview of the main activities that will be carried out.

\subsection{Activities}\label{subsec:activities}

\meta{
\paragraph{Deployment via pulverisation}
The first fundamental step of this project is to provide an implementation for
the pulverisation approach via a dedicated framework.
%
The framework will be designed to be modular and extensible to allow the integration
of different technologies and give the possibility to be used in conjunction with
\ac{ac} frameworks.

\paragraph{Dynamic reconfiguration and relocation}
Another key contribution of this project is to support dynamic reconfiguration of the system,
allowing it to adapt to the changes in the infrastructure and external conditions.
%
In the original model of the pulverisation approach,
there is no support for dynamic reconfiguration of the system.
%
For this reason,
the project will investigate the extension of the pulverisation approach
to support dynamic reconfiguration and relocation of the system inside the framework stack.
%
To do so,
the project will leverage machine learning techniques to manage the complex task of reconfiguration
to opportunistically balance the performance and energy consumption or any other system-specific \ac{qos}.

\paragraph{Bind pulverisation to AC frameworks}
The need to integrate \ac{ac} framework into the pulverisation one is
a fundamental step to provide a complete solution for the design and deployment of
collective adaptive applications on heterogeneous infrastructures.
%
Indeed, combining the pulverisation approach with \ac{ac} frameworks
allows to describe the system's behavior at a macro-level
but at the same time exploiting the pulverisation to
effectively deploy the system on the available infrastructure,
managing also the dynamic reconfiguration of the system transparently and seamlessly.

\paragraph{Big-data for Aggregate Computing}
???
}

\subsection{Scope}\label{subsec:scope}

\subsection{Technology}\label{subsec:technology}

\meta{
\subsubsection*{Cloud computing}
\subsubsection*{Big-data}
\subsubsection*{Distributed technology}
}

\section{Expected results}\label{sec:expected-results}

\end{document}
